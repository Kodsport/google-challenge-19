\problemname{Coin Counter}
Given an image of a set of Swedish coins valued 1, 5 and 10 SEK, what is their sum?

The coins look like this:
\begin{figure}[h]
  \centering
  \includegraphics[width=0.2\textwidth]{1.png}
  \caption{1 SEK coin}
\end{figure}
\begin{figure}[h]
  \centering
  \includegraphics[width=0.2\textwidth]{5.png}
  \caption{5 SEK coin}
\end{figure}
\begin{figure}[h]
  \centering
  \includegraphics[width=0.2\textwidth]{10.png}
  \caption{10 SEK coin}
\end{figure}

\section*{Input}
The input is a JPEG-formatted image.

\section*{Output}
Output a single number; the sum of all the coins in the input image.

\section*{Scoring}
Your solution will be tested on a set of test groups, each worth a number of points.
To get the points for a test group you need to solve all test cases in the test group.
Your final score will be the maximum score of a single submission.

\noindent
\begin{tabular}{| l | l | l |}
\hline
Group & Points & Constraints \\ \hline
1     & 20     & There are only 1 SEK coins. None of the coins are rotated or cover each other. \\ \hline
2     & 30     & There are only 1 SEK coins. None of the coins cover each other. \\ \hline
3     & 75     & There are only 1 SEK coins. \\ \hline
4     & 60     & None of the coins are rotated or cover each other. \\ \hline
5     & 90     & None of the coins cover each other. \\ \hline
6     & 225     & No further restrictions. \\ \hline
\end{tabular}
